\begin{frame}
	\frametitle{Prueba $a, b \in P, a \prec b$ si y solo si $f(a) \prec^{\prime} f(b)$}

	\begin{block}{}
		\begin{enumerate}[2)]
			\item<1-> Supongamos que $\exists z^{\prime}$ tal que $f(a) <^{\prime} z^{\prime} <^{\prime} f(b)$. Luego,
			nuevamente utilizando $(\dag)$, tenemos:
		\end{enumerate}
		\onslide<2->{\begin{eqnarray*}
			f^{-1}(f(a)) &<& f^{-1}(z^{\prime}) < f^{-1}(f(b)) \\
			a &<& f^{-1}(z^{\prime}) < b
		\end{eqnarray*}

		\PN Lo cual, contradice (ii)}. \onslide<3->{El absurdo vino de suponer que $\exists z^{\prime}$ tal que
		$f(a) <^{\prime} z^{\prime} <^{\prime} f(b)$, por lo tanto $\nexists z^{\prime}$ tal que
		$f(a) <^{\prime} z^{\prime} <^{\prime} f(b)$.}

		\PN \onslide<4->{Finalmente, dado que se cumplen los puntos 1) y 2), se cumple también $f(a) \prec^{\prime} f(b)$.}

		\vspace{5mm}
		\PN \onslide<5->{\begin{tabular}{|c|} \hline $\Leftarrow$ \\\hline \end{tabular} Supongamos $f(a) \prec^{\prime}
		f(b)$, veamos que $a \prec b$.}

		\PN \onslide<6->{Ya que $f^{-1}: (P^{\prime}, \leq^{\prime}) \rightarrow (P, \leq)$ es isomorfismo, por lo ya visto
		tenemos:
		\begin{eqnarray*}
			f^{-1}(f(a)) &\prec& f^{-1}(f(b)) \\
			a &\prec& b
		\end{eqnarray*}}
	\end{block}
\end{frame}
